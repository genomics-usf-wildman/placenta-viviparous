% species names
\newcommand{\smonkey}[0]{\textit{A. fusciceps}\xspace}
\newcommand{\cow}[0]{\textit{B. taurus}\xspace}
\newcommand{\dog}[0]{\textit{C. familiaris}\xspace}
\newcommand{\armadillo}[0]{\textit{D. novemcinctus}\xspace}
\newcommand{\horse}[0]{\textit{E. caballus}\xspace}
\newcommand{\elephant}[0]{\textit{L. africana}\xspace}
\newcommand{\human}[0]{\textit{H. sapiens}\xspace}
\newcommand{\opossum}[0]{\textit{M. domestica}\xspace}
\newcommand{\mouse}[0]{\textit{M. musculus}\xspace}
\newcommand{\galili}[0]{\textit{S. galili}\xspace}
\newcommand{\sheep}[0]{\textit{O. aries}\xspace}
\newcommand{\bonobo}[0]{\textit{P. paniscus}\xspace}
\newcommand{\carmeli}[0]{\textit{S. carmeli}\xspace}
\newcommand{\pig}[0]{\textit{S. scrofa}\xspace}

% top 50 gene table explanation. Note that if you modify this, you
% should probably also modify the human table explanation, which is
% slightly different.
\newcommand{\tableexplanation}[2]{The top 50 genes by FPKM in #1 which
  are not annotated as Mt\_rRNA, Mt\_tRNA, rRNA or whose \human
  ortholog is a housekeeping gene. Gene symbol is the gene symbol or
  Ensembl gene id. Human Symbol is the 1:1 human ortholog or the best
  human alignment when enclosed in parenthesis. Human Name is the HUGO
  name corresponding to the human symbol. A complete list of all genes
  is in \cref{#2}.}

\newcommand{\genetreeexplanation}[2][1]{Gene tree showing expression
  and ancestral expression of genes related to #2. The gene tree was
  obtained from Ensembl~\cite{Cunningham.ea2015:Ensembl2015} and genes
  from species not included in this study were excluded from the
  plotted tree. The coloring of the internal nodes of the plots
  indicates the ancestral expression as estimated by fastAnc using
  phylotools. Genes with duplicate gene names in the same species (or
  no gene named defined) are listed by the Ensemble gene ID with
  leading zeros elided.}

\newcommand{\lineagetableexplanation}[3]%
[A two-sided t test was used to calculate p values.]{%
  Genes whose expression changed in #2 as compared to #3. #1 FDR
  is the False Discovery Rate as estimated by the Benjamini and
  Hochberg procedure \cite{Benjamini1995:bh_procedure}.
  $\mathrm{log}_2$ FC is the base-2 logarithm of the fold change from
  #2 to #3 .}

\newcommand{\lineagetableexplanationtypeiii}[2]{%
  \lineagetableexplanation[A nested type-III anova was used to
  calculate p values, and the top 40 genes by $\text{log}_2$
  Fold-Change which changed significantly (FDR $\le 0.05$) are
  shown. ]{#1}{#2}}



%%% Local Variables:
%%% mode: latex
%%% TeX-master: "manuscript.Rnw"
%%% End:
